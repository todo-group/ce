% -*- coding: utf-8 -*-

\documentclass[10pt,dvipdfmx]{beamer}
\usepackage{tutorial}

\title{計算機実験I (第1回)}
\date{2024/04/10}

\begin{document}

\begin{frame}
  \titlepage
  \tableofcontents
\end{frame}

\section{講義・実習の概要}

\begin{frame}[t]{講義・実習の目的}
  \begin{itemize}
    %\setlength{\itemsep}{1em}
  \item 理論・実験を問わず、学部〜大学院〜で必要となる現代的かつ普遍的な計算機の素養を身につける
  \item UNIX環境に慣れる(シェル、ファイル操作、エディタ)
  \item ネットワークの活用 (リモートログイン、共同作業)
  \item プログラムの作成(C言語、コンパイラ、プログラム実行)
  \item 基本的な数値計算アルゴリズム・数値計算の常識を学ぶ
  \item 科学技術文書作成に慣れる(\LaTeX, グラフ作成)
  \end{itemize}
\end{frame}

\begin{frame}[t]{身に付けて欲しいこと}
  \begin{itemize}
    %\setlength{\itemsep}{1em}
  \item ツールとしてないものは自分で作る (物理の伝統)
  \item すでにあるものは積極的に再利用する (車輪の再発明をしない)
  \item 数学公式と数値計算アルゴリズムは別物
  \item 刻み幅・近似度合いを変えて何度か計算を行う
  \item グラフ化して目で見てみる
  \item 計算量(コスト)のスケーリング(次数)に気をつける
  \item (計算機は指示したことを指示したようにしかやってくれないということを認識する)
  \end{itemize}
\end{frame}

\begin{frame}[t]{講義・実習内容}
  \begin{itemize}
    % \setlength{\itemsep}{1em}
  \item UNIX操作・ネットワーク
  \item プログラミング: C言語、数値計算ライブラリの利用
  \item ツール: エディタ、コンパイラ、\LaTeX、gnuplot
  \item 数値計算の基礎
  \item 常微分方程式の解法
  \item 連立一次方程式の解法
  \item 行列の対角化
  \item 線形回帰
  \end{itemize}
\end{frame}

\begin{frame}[t,fragile]{スタッフ・進め方・評価}
  \begin{itemize}
    %\setlength{\itemsep}{1em}
  \item スタッフ \href{mailto:computer@exa.phys.s.u-tokyo.ac.jp}{computer@exa.phys.s.u-tokyo.ac.jp}
    \begin{itemize}
    \item 講義: 藤堂
    \item 実習: 高橋助教、山崎助教
    \item 実習TA: 磯貝、菅原
    \end{itemize}
  \item 講義・実習の進め方
    \begin{itemize}
    \item 講義(座学)と実習を交互に実施
    \item 実習の回には各自PCを持参すること
    \end{itemize}
  \item 評価
    \begin{itemize}
    \item 出席(ITC-LMSでのアンケートに回答)

      講義・実習当日11:00-13:00の間に回答
      
    \item レポート(計3回)
    \end{itemize}    
  \end{itemize}    
\end{frame}

\begin{frame}[t]{講義日程(予定)}
  \begin{itemize}
    % \setlength{\itemsep}{1em}
  \item 全8回 (水曜2限 10:25-12:10)
    \begin{itemize}
    \item 4月5日 講義1: 講義の概要・基本的なアルゴリズム
    \item 4月19日 演習1: 環境整備・C言語プログラミング・図のプロット \item 4月26日 講義2: 常微分方程式
    \item 5月10日 演習2 (グループ1): 基本的なアルゴリズム・常微分方程式
    \item 5月17日 演習2 (グループ2): 基本的なアルゴリズム・常微分方程式
    \item {\color{gray} 5月24日 休講}
    \item {\color{gray} 5月31日 休講}
    \item 6月7日 講義3: 連立方程式
    \item 6月14日 演習3 (グループ1): 連立方程式
    \item 6月21日 演習3 (グループ2): 連立方程式
    \item {\color{gray} 6月28日 休講}
    \item 7月5日 講義4: 行列の対角化
    \item 7月12日 演習4 (グループ1): 行列の対角化
    \item 7月19日 演習4 (グループ2): 行列の対角化
    \end{itemize}
  \item 2回目以降の演習はクラスを2グループに分けて実施(グループ1: 学生証番号が奇数、グループ2: 偶数)
  \end{itemize}
\end{frame}

\begin{frame}[t,fragile]{レポート(予定)}
  \begin{itemize}
    %\setlength{\itemsep}{1em}
  \item レポート
    \begin{itemize}
    \item 各自が \LaTeX で作成の上提出 (計3回)
    \item 提出方法: ITC-LMSでPDFファイルを提出
    \end{itemize}
  \item レポートNo.1
    \begin{itemize}
    \item \mbox{} [数値誤差・アルゴリズム基礎] 2題、[常微分方程式] うちから1題の合計3題を選択
    \item 締切: 5月31日
    \end{itemize}
  \item レポートNo.2
    \begin{itemize}
    \item \mbox{} [連立一次方程式]から2題を選択
    \item 締切: 7月5日
    \end{itemize}
  \item レポートNo.3
    \begin{itemize}
    \item \mbox{} [対角化]から2題を選択
    \item 締切: 8月2日
    \end{itemize}
  \item 詳細はITC-LMSに掲示予定
  \end{itemize}    
\end{frame}

\begin{frame}[t]{講義資料}
  \begin{itemize}
    % \setlength{\itemsep}{1em}
  \item 講義資料置き場:
  
    {\footnotesize \href{https://github.com/todo-group/ComputerExperiments/tree/2023s-computer1}{https://github.com/todo-group/ComputerExperiments/tree/2023s-computer1}}
    \begin{itemize}
    \item 計算機実験ハンドブック
      \begin{itemize}
      \item UNIX入門
      \item gnuplot入門
      \item C言語入門
      \item \LaTeX 入門
      \end{itemize}
    \item \href{https://utphys-comp.github.io}{計算機実験のための環境整備}({\small \href{https://utphys-comp.github.io}{https://utphys-comp.github.io}})
    \item 講義資料
    \item 実習課題
    \item サンプルコード
    \end{itemize}
  \end{itemize}
\end{frame}

\begin{frame}[t]{質問がある場合には…}
  \begin{itemize}
    %\setlength{\itemsep}{1em}
  \item 講義・実習時間中は自由に質問してください
    \begin{itemize}
    \item 計算機実験Slackでこっそり質問も可
    \end{itemize}
  \item 講義時間外も質問を受け付けます
    \begin{itemize}
    \item 計算機実験Slack (参加方法はITC-LMSの掲示板参照)
    \item ITC-LMS「担当教員へのメッセージ」
    \item メール(\href{mailto:computer@exa.phys.s.u-tokyo.ac.jp}{computer@exa.phys.s.u-tokyo.ac.jp})
    \end{itemize}
  \item メール・Slackなどで質問するときに注意すべきこと
    \begin{itemize}
    \item (メールの場合) Subjectをきちんとつける、きちんと名乗る
    \item 実行環境を明示する
    \item 問題を再現する手順を明記する
    \item 関連するファイル(Cや \LaTeX のソースコード等)を添付する
    \item エラーメッセージを添付する
    \end{itemize}
  \end{itemize}
\end{frame}

  

\input{0_about/env.tex}
\input{1_basics/error.tex}
\input{1_basics/diff-01.tex}
\input{1_basics/diff-02.tex}

\section{ニュートン法}
\input{1_basics/newton-01.tex}
\input{1_basics/newton-02.tex}
\input{1_basics/newton-03.tex}
\begin{frame}[t,fragile]{ニュートン法による最適化}
  \begin{itemize}
    \setlength{\itemsep}{1em}
  \item $x$は$d$次元のベクトル: $x = {}^t(x_1,x_2,\cdots,x_d)$、目的関数$f(x)$はスカラー
  \item 勾配ベクトル: $\displaystyle [\nabla f(x)]_i = \frac{\partial f(x)}{\partial x_i}$
  \item 極小値(最小値)となる条件: $\nabla f(x)=0$
  \item ニュートン法で$f(x)$を$\nabla f(x)$で置き換えればよい
  \item 次の解の候補: $\displaystyle x_{n+1} = x_n - H^{-1}(x_n) \nabla f(x_n)$
  \item ヘッセ行列(Hessian): $\displaystyle H_{ij}(x) = \frac{\partial^2 f}{\partial x_i \partial x_j}(x)$
  \end{itemize}
\end{frame}

\begin{frame}[t,fragile]{準ニュートン法}
  \begin{itemize}
    %\setlength{\itemsep}{1em}
  % \item ニュートン法では、ヘッセ行列の計算・保存が必要
  \item 準ニュートン法: それまでの反復で計算した勾配ベクトルから、ヘッセ行列の逆行列を近似($C_n \approx H^{-1}$)
  \begin{itemize}
    \item 極小点の近傍では
      \begin{align*}
          x-x_n &= -H^{-1} \cdot \nabla f(x_n) \\
          x-x_{n+1} &= -H^{-1} \cdot \nabla f(x_{n+1})
      \end{align*}
    \item 差をとると($s_n = x_{n+1} - x_n$, $y_n = \nabla f(x_{n+1}) - \nabla f(x_n)$)
      \begin{align*}
        s_n &= H^{-1} \cdot y_n
      \end{align*}
    \item この式が満たされるように、$C_{n+1}$を$C_{n}+(\text{補正})$の形で構成(DFP法: Davidon-Fletcher-Powell)
      \[
        C_{n+1} = C_{n} + \frac{s_n s_n^T}{s_n^T y_n} - \frac{C_{n} y_n (C_{n} y_n)^T}{y_n^T C_n y_n}
      \]
    \item 他にも、BFGS法など
    \end{itemize}
  \end{itemize}
\end{frame}

\input{7_optimization/descent-01.tex}
\input{7_optimization/descent-02.tex}
\input{1_basics/newton-06.tex}
\begin{frame}[t,fragile]{反復計算}
  \begin{itemize}
    %\setlength{\itemsep}{1em}
  \item {\tt while}による反復 (ハンドブック2.2.5節)
\begin{lstlisting}
double residual = 1;    /* 残差: 適当に大きな値 */
double error = 1;       /* 誤差: 適当に大きな値 */
double delta = 1.0e-12; /* 欲しい精度 */
while (residual > delta && error > delta) {
  /* ニュートン法の漸化式 */
  /* residual と error を計算 */
}
\end{lstlisting}
残差と誤差のどちらかが欲しい精度に達したら計算を終了
\item {\tt break}を使う例 (ハンドブック2.2.6節)
\begin{lstlisting}
for (;;) {
  /* ニュートン法の漸化式 */
  /* residual と error を計算 */
  if (residual < delta || error < delta) break;
}
\end{lstlisting}
  \end{itemize}
\end{frame}

\input{1_basics/newton-08.tex}

\input{1_basics/bisection.tex}
\input{7_optimization/bracketing.tex}

\section{}
\begin{frame}[t]{おわりに}
  \begin{itemize}
    % \setlength{\itemsep}{1em}
  \item 出席
  \begin{itemize}
    \item UTOLでのアンケートに回答してください(期限は本日15:00)
  \end{itemize}
  \item 次回
    \begin{itemize}
    \item 4月17日 実習1 (グループ1): 環境整備・C言語プログラミング
    \item 4月24日 実習1 (グループ2): 環境整備・C言語プログラミング
    \item グループ1: 学生証番号が奇数、グループ2: 偶数
    \item ノートPCを持参してください
    \end{itemize}
  \end{itemize}
\end{frame}

\end{document}
