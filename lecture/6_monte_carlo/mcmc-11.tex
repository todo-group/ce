%-*- coding:utf-8 -*-

\begin{frame}[t,fragile]{マルコフ連鎖モンテカルロ法}
  \begin{itemize}
    %\setlength{\itemsep}{1em}
  \item 任意の多次元確率分布関数について、その分布したがうサンプルを生成する方法
    \begin{itemize}
    \item 分布関数は正規化されている必要はない
    \item 逆に、正規化定数をマルコフ連鎖モンテカルロで計算するのは難しい
    \end{itemize}
  \item 統計誤差はサンプルの生の分散$s^2$とサンプル数$M$、自己相関時間$\tau$で決まる
    \[
    \sigma^2 \simeq \frac{s^2 (1+2\tau)}{M}
    \]
    \begin{itemize}
    \item 一度に大きく配位を動かそうとすると棄却率が増加 $\Rightarrow$ $\tau$が増加
    \item 動かす幅を小さくするとは採択率は高いが相関が消えない $\Rightarrow$ $\tau$が増加
    \item 非局所更新法、拡張アンサンブル法など様々な方法が使われている
    \end{itemize}
  \item 物理以外でも、Bayes推定や機械学習、社会現象のシミュレーションなど広く使われている
  \end{itemize}
\end{frame}
