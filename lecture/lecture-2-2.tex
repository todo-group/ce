%-*- coding:utf-8 -*-

\documentclass[10pt,dvipdfmx]{beamer}
\usepackage{tutorial}

\title{計算機実験II (L2) --- 偏微分方程式と多体系の量子力学}
\date{2023/10/27}

\begin{document}

\begin{frame}
  \titlepage
  \tableofcontents
\end{frame}

\begin{frame}[t]{講義日程 (予定)}
  \begin{itemize}
    % \setlength{\itemsep}{1em}
  \item 全8回 (金曜5限 16:50-18:35)
    \begin{itemize}
    \item 10月4日(金) 講義1: 多体系の統計力学とモンテカルロ法
    \item {\color{gray} 10月11日(金) 休講 (物理学教室コロキウム)}
    \item 10月18日(金) 実習1
    \item {\color{gray} 10月25日(金) 休講}
    \item 11月1日(金) 講義2: 偏微分方程式と多体系の量子力学
    \item 11月8日(金) 実習2
    \item {\color{gray} 11月15日(金) 休講}
    \item 11月29日(金) 講義3: 少数多体系・分子動力学
    \item 12月6日(金) 実習3
    \item {\color{gray} 12月13日(金) 休講 (物理学教室コロキウム)}
    \item {\color{gray} 12月20日(金) 休講 (ニュートン祭)}
    \item 12月27日(金) 講義4: 最適化問題
    \item 1月10日(金) 実習4
    \item {\color{gray} 1月24日(金) 休講 (物理学教室コロキウム)}
    \end{itemize}
  \end{itemize}
\end{frame}


\section{偏微分方程式の初期値問題}

\input{2_ode/ode-02.tex}
\input{2_ode/pde-01.tex}
\begin{frame}[t]{有限差分法}
  \begin{itemize}
  \item $t$に関して前進差分を考える
    \[
    \frac{\partial u}{\partial t} \Big|_{(j \Delta x, n \Delta t)} = \frac{u_j^{n+1} - u_j^n}{\Delta t} + {\cal O}(\Delta t)
    \]
  \item $x$に関しては中心差分を考える
    \[
    \frac{\partial^2 u}{\partial x^2} \Big|_{(j \Delta x, n \Delta t)} = \frac{u_{j+1}^{n} - 2 u_{j}^{n} + u_{j-1}^{n}}{\Delta x^2} + {\cal O}(\Delta x^2)
    \]
  \item 拡散方程式に代入して整理すると
    \[
    u_{j}^{n+1} = u_{j}^{n} + r (u_{j+1}^{n} - 2 u_{j}^{n} + u_{j-1}^{n}) + \Delta t q_{j}^{n} \qquad (r = \frac{D\Delta t}{\Delta x^2})
    \]
  \item FTCS (Forward-Time Centered Space)法
  \end{itemize}
\end{frame}

\input{2_ode/pde-03.tex}
\begin{frame}[t]{有限差分法の安定性}
  \begin{itemize}
  \item (陽的)有限差分法においては、$\Delta t$、$\Delta x$は小さければ小さいほどよいというわけではない
    \begin{align*}
      r = \frac{D \Delta t}{\Delta x^2}
    \end{align*}
  \item 一次元拡散方程式の場合
    \begin{align*}
      \begin{cases}
        r \le 1/2 & \text{安定} \\
        r > 1/2 & \text{\color{red}不安定}
      \end{cases}
    \end{align*}
  \item $\Delta x$を半分にしたら、$\Delta t$は1/4にしなければならない

    $\Rightarrow$ 計算量は8倍
  \end{itemize}
\end{frame}

\begin{frame}[t]{Von Neumannの安定性解析}
  \begin{itemize}
  \item $u(x,t)$のフーリエ変換を導入する
    \begin{align*}
    u(x,t) &= \sum_k v(k,t) e^{ikx} \\
    u_j^n &= \sum_k v_k^n e^{ik\Delta x \cdot j}
    \end{align*}
  \item FTCS法の式に代入し、$k$の項を取り出すと
    \[
    v_{k}^{n+1} =  (1 + 2r(\cos k \Delta x - 1)) v_{k}^{n}
    \]
  \item 全ての$k$に対し発散しない(=係数の絶対値が1未満になる)ためには
    \[
      | 1 - 4r | < 0 \qquad \text{すなわち} \ r < \frac{1}{2}
    \]
  \end{itemize}
\end{frame}

\begin{frame}[t]{一次元波動方程式(双極型)}
  \begin{itemize}
  \item 一次元波動方程式
    \[
    \frac{\partial^2 u}{\partial t^2} = c^2 \frac{\partial^2 u}{\partial x^2} \qquad u(x,0)=f(x), \frac{\partial u}{\partial t} (x,0) = g(x)
    \]
  \item $t$に関する中心差分
    \[
    \frac{\partial^2 u}{\partial t^2} \Big|_{(j \Delta x, n \Delta t)} = \frac{u_{j}^{n+1} - 2 u_{j}^{n} + u_{j}^{n-1}}{\Delta t^2} + {\cal O}(\Delta t^2)
    \]
  \item 代入して整理すると
    \[
    u_{j}^{n+1} = 2u_{j}^{n} - u_{j}^{n-1} + \alpha^2 (u_{j+1}^{n} - 2 u_{j}^{n} + u_{j-1}^{n}) \qquad (\alpha = c\frac{\Delta t}{\Delta x})
    \]
  \item 解が安定であるための条件
    \[
    \alpha = c\frac{\Delta t}{\Delta x} \le 1
    \]
  \end{itemize}
\end{frame}

\input{2_ode/pde-06.tex}
\input{2_ode/pde-07.tex}
\input{2_ode/pde-08.tex}
\input{2_ode/pde-09.tex}
\input{2_ode/pde-10.tex}
\input{2_ode/pde-11.tex}

% \section{対角化 (復習)}
% \input{4_eigenvalue_problem/intro-03.tex}
% \input{4_eigenvalue_problem/intro-04.tex}
% \input{4_eigenvalue_problem/intro-07.tex}

\section{横磁場イジング模型}

\input{4_eigenvalue_problem/tfi-01.tex}
\input{4_eigenvalue_problem/tfi-02.tex}
\input{4_eigenvalue_problem/tfi-03.tex}

\section{多体量子系の時間発展}

\input{4_eigenvalue_problem/tfi-04.tex}
\input{4_eigenvalue_problem/tfi-05.tex}
\input{4_eigenvalue_problem/tfi-06.tex}

\section{量子コンピュータ}

\input{4_eigenvalue_problem/tfi-07.tex}
\input{4_eigenvalue_problem/tfi-08.tex}
\begin{frame}[t,fragile]{量子加算器}
  \begin{itemize}
    %\setlength{\itemsep}{1em}
  \item 1量子ビットの加算 $\Rightarrow$ CCXゲートとCXゲートで実現できる
    \begin{center}
      \resizebox{0.5\textwidth}{!}{\includegraphics{image/adder1.pdf}}
    \end{center}
  \item CCXゲートは、Hゲート、Rzゲート、CXゲートの組み合わせで表現できる
  \item 任意のユニタリ変換は、Hゲート、Rzゲート、CXゲートの組み合わせで表現できる (万能量子ゲート)
  \end{itemize}
\end{frame}

\input{4_eigenvalue_problem/tfi-10.tex}

\end{document}
