\section{講義・実習の概要}

\begin{frame}[t]{講義・実習の目的}
  \begin{itemize}
    %\setlength{\itemsep}{1em}
  \item 理論・実験を問わず、学部〜大学院〜で必要となる現代的かつ普遍的な計算機の素養を身につける
  \item {\color{gray}UNIX環境に慣れる(シェル、ファイル操作、エディタ)}
  \item {\color{gray}ネットワークの活用 (リモートログイン、共同作業)}
  \item {\color{gray}プログラムの作成(C言語、コンパイラ、プログラム実行)}
  \item 基本的な数値計算アルゴリズム・数値計算の常識を学ぶ
  \item {\color{gray}科学技術文書作成に慣れる(\LaTeX, グラフ作成)}
  \item {\color{red}物理学における具体的な問題を通して実践的な知識と経験を身につける}
  \end{itemize}
\end{frame}

\begin{frame}[t]{身に付けて欲しいこと}
  \begin{itemize}
    %\setlength{\itemsep}{1em}
  \item ツールとしてないものは自分で作る (物理の伝統)
  \item すでにあるものは積極的に再利用する (車輪の再発明をしない)
  \item 数学公式と数値計算アルゴリズムは別物
  \item 刻み幅・近似度合いを変えて何度か計算を行う
  \item グラフ化して目で見てみる
  \item 計算量(コスト)のスケーリング(次数)に気をつける
  \item (計算機は指示したことを指示したようにしかやってくれないということを認識する)
  \item {\color{red}問題の解き方は一通りではない}
  \item {\color{red}いろいろな手法を組み合わせて使う}
  \end{itemize}
\end{frame}

\begin{frame}[t]{講義・実習内容}
  \begin{itemize}
    \setlength{\itemsep}{1em}
  \item 問題解決型: 計算機実験Iで身に付けた知識をもとに、より高度な数値計算手法・アルゴリズムを学び、物理学における具体的な問題への応用を通して実践的な知識と経験を身につける
    \begin{itemize}
    \item 取り上げる問題
      \begin{itemize}
      \item ランダムプロセス
      \item 統計力学(イジング模型)
      \item 拡散方程式、波動方程式
      \item 量子力学(1粒子系、実時間発展、横磁場イジング模型、量子コンピュータ)
      \item 力学系・粒子系
        
      \end{itemize}
    \item 取り上げる計算手法・アルゴリズム
      \begin{itemize}
      \item モンテカルロ法・マルコフ連鎖モンテカルロ法
      \item 行列の対角化・疎行列分解
      \item 有限差分法、偏微分方程式の解法
      \item 常微分方程式の積分、分子動力学
      \item 連続最適化、離散最適化
      \end{itemize}
    \end{itemize}
    などを予定
  \end{itemize}
\end{frame}

\begin{frame}[t]{講義日程 (予定)}
  \begin{itemize}
    % \setlength{\itemsep}{1em}
  \item 全8回 (金曜5限 16:50-18:35)
    \begin{itemize}
    \item 10月4日(金) 講義1: 多体系の統計力学とモンテカルロ法
    \item {\color{gray} 10月11日(金) 休講 (物理学教室コロキウム)}
    \item 10月18日(金) 実習1
    \item {\color{gray} 10月25日(金) 休講}
    \item 11月1日(金) 講義2: 偏微分方程式と多体系の量子力学
    \item 11月8日(金) 実習2
    \item {\color{gray} 11月15日(金) 休講}
    \item 11月29日(金) 講義3: 少数多体系・分子動力学
    \item 12月6日(金) 実習3
    \item {\color{gray} 12月13日(金) 休講 (物理学教室コロキウム)}
    \item {\color{gray} 12月20日(金) 休講 (ニュートン祭)}
    \item 12月27日(金) 講義4: 最適化問題
    \item 1月10日(金) 実習4
    \item {\color{gray} 1月24日(金) 休講 (物理学教室コロキウム)}
    \end{itemize}
  \end{itemize}
\end{frame}


\begin{frame}[t,fragile]{講義と実習}
  \begin{itemize}
    %\setlength{\itemsep}{1em}
  \item スタッフ \href{mailto:computer@phys.s.u-tokyo.ac.jp}{computer@phys.s.u-tokyo.ac.jp}
    \begin{itemize}
    \item 講義: 藤堂
    \item 実習: 高橋助教、山崎助教
    \item 実習TA: 高波
    \end{itemize}
  \item 講義・実習の進め方
    \begin{itemize}
    \item 講義(座学)と実習を実施
    \item 実習の回には各自PCを持参すること
    \end{itemize}
  \item 評価
    \begin{itemize}
    \item 出席(ITC-LMSでのアンケートに回答)

      講義・実習当日16:50-20:00の間に回答
      
    \item レポート (計2回)
    \end{itemize}    
  \end{itemize}    
\end{frame}
