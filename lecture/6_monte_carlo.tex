% -*- coding: utf-8 -*-

\documentclass[10pt,dvipdfmx]{beamer}
\usepackage{tutorial}
\title{計算機実験 --- モンテカルロ法}

\begin{document}

\lstset{language={C},basicstyle=\ttfamily\scriptsize,showspaces=false,rulecolor=\color[cmyk]{0, 0.29,0.84,0}}

\begin{frame}
  \titlepage
  \tableofcontents
\end{frame}

\input{6_monte_carlo/statphys.tex}
\input{6_monte_carlo/counting.tex}
\input{6_monte_carlo/transfer.tex}
\input{6_monte_carlo/randomized.tex}
\input{6_monte_carlo/process.tex}
\input{6_monte_carlo/rng.tex}
\input{6_monte_carlo/histogram.tex}
\input{6_monte_carlo/integration.tex}
\input{6_monte_carlo/mcmc.tex}
\input{6_monte_carlo/molecular_dynamics.tex}
\input{6_monte_carlo/longrange.tex}
\input{6_monte_carlo/virial.tex}
% -*- coding: utf-8 -*-

\section{ハミルトニアン・モンテカルロ法}

%-*- coding:utf-8 -*-

\begin{frame}[t,fragile]{ハミルトニアン・モンテカルロ法}
  \begin{itemize}
    %\setlength{\itemsep}{1em}
  \item 通常のマルコフ連鎖モンテカルロ法
  \begin{itemize}
    \item 配位を局所的に変化させる ⇒ 配位空間上でランダムウォーク(拡散)
    \item 配位をむやみに大きく変化させようとしてもかえって状況は悪化
    \begin{itemize}
      \item cf. クラスターアルゴリズム, 拡張アンサンブル法
    \end{itemize}
  \end{itemize}

  \item ハミルトニアン・モンテカルロ法(HMC: Hamiltonian Monte Carlo)
  \begin{itemize}
    \item 別名: ハイブリッド・モンテカルロ法(Hybrid Monte Carlo)
    \item 分子動力学法 + メトロポリス法
    \item 仮想的な運動量を導入し、拡張した位相空間上でのマルコフ連鎖を考える
    \begin{itemize}
      \item 分子動力学法による仮想的な時間発展を用いて次の候補配位を生成
      \item 高い acceptance rate を保ちながら、大域的な配位変化を実現
    \end{itemize}
    \item 物性物理ではあまり使われていないが、Lattice QCDやベイズ推定ではよく使われている
    \end{itemize}
\end{itemize}
\end{frame}

%-*- coding:utf-8 -*-

\begin{frame}[t,fragile]{ハミルトニアン・モンテカルロ法が有効な例}
  \begin{itemize}
    %\setlength{\itemsep}{1em}
  \item 状態変数が連続変数の場合のみ使用可能
  \begin{itemize}
    \item 結合した調和振動子系
      \begin{align*}
        E(x_1,x_2,\cdots,x_N) = \sum x_i^2 + \sum_{i,j} C_{i,j} (x_i - x_j)^2
      \end{align*}
    \item 連続古典スピン系 (ハイゼンベルグ模型)
      \begin{align*}
        E(\vec{S}_1, \vec{S}_2, \cdots, \vec{S}_N) = \sum_{i,j} J_{i,j} \vec{S}_i \cdot \vec{S}_j
      \end{align*}
  \end{itemize}

  \item エネルギーの状態変数に関する微分の情報(=力)を利用する
  \item 現実の系の時間発展と同一である必要はない - 仮想的な時間発展
\end{itemize}
\end{frame}

%-*- coding:utf-8 -*-

\begin{frame}[t,fragile]{ハミルトン力学系}
  \begin{itemize}
    %\setlength{\itemsep}{1em}
  \item もともとの状態変数: \(\vec{q} = (q_1,q_2,\cdots,q_N)\)
  \item ボルツマン重み(逆温度\(\beta\)は\(E\)に含む): \(P(\vec{q}) = e^{-E(\vec{q})}\)
  \item 状態変数の「時間発展」を考える
  \item 「運動量」「運動エネルギー」「ハミルトニアン」を導入
    \begin{align*}
      \vec{p} &= \frac{d\vec{q}}{dt} = (\vec{p}_1(t),\vec{p}_2(t),\cdots,\vec{p}_N(t)) \\
      H(\vec{q}, \vec{p}) &= K(\vec{p}) + E(\vec{q}) = \frac{1}{2}\sum_i p_i^2 + E(\vec{q})
    \end{align*}
  \item 「ハミルトン力学系」の「時間発展」
    \begin{align*}
      \begin{cases}
        \displaystyle
        \frac{dq_i}{dt} = \frac{\partial H}{\partial p_i} \\
        \displaystyle
        \frac{dp_i}{dt} = -\frac{\partial H}{\partial q_i}
      \end{cases}
    \end{align*}
\end{itemize}
\end{frame}

%-*- coding:utf-8 -*-

\begin{frame}[t,fragile]{ハミルトン力学系}
  \begin{itemize}
    %\setlength{\itemsep}{1em}
  \item 「時間発展」の特徴
  \begin{itemize}
    \item 「全エネルギー」が保存
    \[
      \frac{dH}{dt} = \frac{\partial H}{\partial q} \frac{dq}{dt} + \frac{\partial H}{\partial p} \frac{dp}{dt} = 0
    \]
    \item 位相空間の体積が保存(Liouvilleの定理)

          位相空間上の流れの場\(\bm{v} = (\frac{dq}{dt},\frac{dp}{dt})\)について
          \[
          \text{div} \, \bm{v} = \frac{\partial}{\partial q} \frac{dq}{dt} + \frac{\partial}{\partial p} \frac{dp}{dt} = 0
          \]
          \end{itemize}
  \item 位相空間上の同時分布が\(P(\vec{q},\vec{p}) \sim \exp(-H(\vec{q},\vec{p}))\)形の時
  \begin{itemize}
    \item \(P(\vec{q},\vec{p})\)は時間発展の下で不変
    \item 我々の欲しいボルツマン分布は\(P(\vec{q},\vec{p})\)の周辺分布
  \end{itemize}
\end{itemize}
\end{frame}

%-*- coding:utf-8 -*-

\begin{frame}[t,fragile]{位相空間での状態更新}
  \begin{itemize}
    %\setlength{\itemsep}{1em}
  \item ハミルトン方程式に従って適当な時間\(\tau\)だけ時間発展を行う
  \item 位置\(\vec{q}\)を固定して運動量\(\vec{p}\)だけを確率的に更新 (「全エネルギー」を変更 )
  \begin{itemize}
    \item \(p_i\)の条件付き確率 (正規分布 \(\mathcal{N}(0,1)\))
    \begin{align*}
      p(p_i|\vec{q},p_1,\cdots,p_{i-1},p_{i+1},\cdots,p_N) &\sim \exp (-\frac{1}{2}\sum p_j^2+E(\vec{q})) \\ &\sim \exp(-\frac{1}{2}p_i^2)
    \end{align*}
  \end{itemize}

  \item 実際には差分法を用いて運動方程式を積分するので有限の離散誤差
  \begin{itemize}
    \item Liouvilleの定理を厳密に満たす差分法(リープフロッグ)を用いる
    \item 分子動力学法 + メトロポリス法
    \item 「全エネルギー」の値がずれた分をメトロポリス法で修正
  \end{itemize}

  \item 詳細ついあい条件
  \begin{align*}
    &\exp(-E(\vec{q})) \exp(-K(\vec{p})) P((\vec{q},\vec{p})\rightarrow(\vec{q'},\vec{p'})) \min (1, \exp(H-H')) \\
    & = \exp(-E(\vec{q'})) \exp(-K(\vec{p'})) P((\vec{q'},\vec{p'})\rightarrow(\vec{q},\vec{p})) \min (1, \exp(H'-H))
  \end{align*}
\end{itemize}
\end{frame}

%-*- coding:utf-8 -*-

\begin{frame}[t,fragile]{ハミルトニアン・モンテカルロ法}
  \begin{itemize}
    %\setlength{\itemsep}{1em}
    \item 分子動力学法 + メトロポリス法
    \item 仮想的な運動量を導入し、拡張した位相空間上でのマルコフ連鎖を考える
    \item 分子動力学法による仮想的な時間発展を用いて次の候補配位を生成
    \begin{itemize}
      \item 位相空間の体積を保存する差分法を用いることにより、提案確率が対称になる
    \end{itemize}
    \item 高い acceptance rate を保ちながら、大域的な配位変化を実現
    \begin{itemize}
      \item エネルギーをさらに良く保存する高い次数の差分法を用いれば、より acceptance rate が高くなる
    \end{itemize}
    \item すばやくかき混ぜるというアルゴリズム的な目標を達成するために、仮想的に「物理」を導入
\end{itemize}
\end{frame}


\input{6_monte_carlo/exercise.tex}

\end{document}
