%-*- coding:utf-8 -*-

\documentclass[10pt,dvipdfmx]{beamer}
\usepackage{tutorial}

\title{計算機実験II (L4) --- 最適化問題}
\date{2022/12/23}

\begin{document}

\begin{frame}
  \titlepage
  \tableofcontents
\end{frame}

\begin{frame}[t]{講義日程 (予定)}
  \begin{itemize}
    % \setlength{\itemsep}{1em}
  \item 全8回 (金曜5限 16:50-18:35)
    \begin{itemize}
    \item 10月4日(金) 講義1: 多体系の統計力学とモンテカルロ法
    \item {\color{gray} 10月11日(金) 休講 (物理学教室コロキウム)}
    \item 10月18日(金) 実習1
    \item {\color{gray} 10月25日(金) 休講}
    \item 11月1日(金) 講義2: 偏微分方程式と多体系の量子力学
    \item 11月8日(金) 実習2
    \item {\color{gray} 11月15日(金) 休講}
    \item 11月29日(金) 講義3: 少数多体系・分子動力学
    \item 12月6日(金) 実習3
    \item {\color{gray} 12月13日(金) 休講 (物理学教室コロキウム)}
    \item {\color{gray} 12月20日(金) 休講 (ニュートン祭)}
    \item 12月27日(金) 講義4: 最適化問題
    \item 1月10日(金) 実習4
    \item {\color{gray} 1月24日(金) 休講 (物理学教室コロキウム)}
    \end{itemize}
  \end{itemize}
\end{frame}


\section{最適化問題}

\input{7_optimization/optimization-01.tex}
\input{7_optimization/optimization-02.tex}

\section{1次元の最適化 (復習)}

\input{1_basics/newton-01.tex}
\input{1_basics/newton-03.tex}
\begin{frame}[t,fragile]{ニュートン法による最適化}
  \begin{itemize}
    \setlength{\itemsep}{1em}
  \item $x$は$d$次元のベクトル: $x = {}^t(x_1,x_2,\cdots,x_d)$、目的関数$f(x)$はスカラー
  \item 勾配ベクトル: $\displaystyle [\nabla f(x)]_i = \frac{\partial f(x)}{\partial x_i}$
  \item 極小値(最小値)となる条件: $\nabla f(x)=0$
  \item ニュートン法で$f(x)$を$\nabla f(x)$で置き換えればよい
  \item 次の解の候補: $\displaystyle x_{n+1} = x_n - H^{-1}(x_n) \nabla f(x_n)$
  \item ヘッセ行列(Hessian): $\displaystyle H_{ij}(x) = \frac{\partial^2 f}{\partial x_i \partial x_j}(x)$
  \end{itemize}
\end{frame}

\input{7_optimization/bracketing-01.tex}
\input{7_optimization/bracketing-02.tex}

\section{最急降下法と勾配降下法}

\input{7_optimization/descent-01.tex}
\input{7_optimization/descent-02.tex}
% \input{7_optimization/descent-03.tex}
% \input{7_optimization/descent-04.tex}

\section{共役勾配法}

\input{7_optimization/cg-01.tex}
\input{7_optimization/cg-02.tex}
\input{7_optimization/cg-03.tex}
\input{7_optimization/cg-04.tex}
\input{7_optimization/cg-05.tex}
\input{7_optimization/cg-06.tex}
\input{7_optimization/cg-07.tex}
\input{7_optimization/cg-08.tex}

\section{勾配の計算}

\input{1_basics/diff-01.tex}
% \input{1_basics/diff-02.tex}
\input{1_basics/diff-03.tex}
\input{1_basics/diff-04.tex}
\input{1_basics/diff-05.tex}
\input{1_basics/diff-06.tex}
\input{1_basics/diff-07.tex}
\input{1_basics/diff-08.tex}
\input{1_basics/diff-09.tex}
\input{1_basics/diff-10.tex}
\input{1_basics/diff-11.tex}
\input{1_basics/diff-12.tex}
\input{1_basics/diff-13.tex}

\section{Nelder-Meadの滑降シンプレックス法}

\input{7_optimization/simplex-01.tex}
\input{7_optimization/simplex-02.tex}
\input{7_optimization/simplex-03.tex}

\section{シミュレーテッドアニーリング}

\input{4_eigenvalue_problem/tfi-05.tex}
\input{4_eigenvalue_problem/tfi-06.tex}
\input{7_optimization/annealing-01.tex}
\input{6_monte_carlo/mcmc-07.tex}
\input{7_optimization/annealing-02.tex}

\section{最適化手法の比較}

\input{7_optimization/comparison-01.tex}
\input{7_optimization/comparison-02.tex}
\input{7_optimization/comparison-03.tex}
\input{7_optimization/comparison-04.tex}
\input{7_optimization/comparison-05.tex}
\input{7_optimization/comparison-06.tex}

\section{}

\begin{frame}[t,fragile]{計算機環境}
  \begin{itemize}
    %\setlength{\itemsep}{1em}
  \item 教育用計算機システム
  \item 知の物理学クラスタ(ai)
    \begin{itemize}
    \item 卒業まで利用可 (希望すれば大学院でも)
    \item バッチシステムを使えば、かなり大規模な計算も可能
    \end{itemize}
  \item 計算機利用・シミュレーションに関する質問は今後も歓迎 {\tt computer@phys.s.u-tokyo.ac.jp}
  \end{itemize}
\end{frame}

\end{document}
