%-*- coding:utf-8 -*-

\documentclass[10pt,dvipdfmx]{beamer}
\usepackage{tutorial}

\title{計算機実験II (L3) --- 少数多体系・分子動力学}
\date{2023/11/17}

\begin{document}

\begin{frame}
  \titlepage
  \tableofcontents
\end{frame}

\begin{frame}[t]{講義日程 (予定)}
  \begin{itemize}
    % \setlength{\itemsep}{1em}
  \item 全8回 (金曜5限 16:50-18:35)
    \begin{itemize}
    \item 10月4日(金) 講義1: 多体系の統計力学とモンテカルロ法
    \item {\color{gray} 10月11日(金) 休講 (物理学教室コロキウム)}
    \item 10月18日(金) 実習1
    \item {\color{gray} 10月25日(金) 休講}
    \item 11月1日(金) 講義2: 偏微分方程式と多体系の量子力学
    \item 11月8日(金) 実習2
    \item {\color{gray} 11月15日(金) 休講}
    \item 11月29日(金) 講義3: 少数多体系・分子動力学
    \item 12月6日(金) 実習3
    \item {\color{gray} 12月13日(金) 休講 (物理学教室コロキウム)}
    \item {\color{gray} 12月20日(金) 休講 (ニュートン祭)}
    \item 12月27日(金) 講義4: 最適化問題
    \item 1月10日(金) 実習4
    \item {\color{gray} 1月24日(金) 休講 (物理学教室コロキウム)}
    \end{itemize}
  \end{itemize}
\end{frame}


\section{少数多体系・分子動力学}

\input{6_monte_carlo/molecular_dynamics-15.tex}
%-*- coding:utf-8 -*-

\begin{frame}[t,fragile]{分子動力学法}
  \begin{itemize}
    \setlength{\itemsep}{1em}
  \item 適当な初期条件から、運動方程式に従って位置と運動量を時間発展させる
    \begin{itemize}
    \item Euler法、Runge-Kutta法、リープ・フロッグ法、速度ベルレ法など
    \item $6N$次元の連立微分方程式
    \end{itemize}
  \item 時間発展に関する物理量の時間平均から平均を評価
    \begin{align*}
      \langle A(p,x) \rangle &= \frac{1}{Z(E)} \int A(p,x) \, \delta(H(p,x)-E) \, dp \, dx \\
      &\simeq \frac{1}{t_{\rm max}} \int_0^{t_{\rm max}} A(p(t),x(t)) \, dt
    \end{align*}
    \begin{itemize}
    \item ハミルトニアンが時間依存しない場合は全エネルギーが保存する→ミクロカノニカル分布
    \end{itemize}
  \item 平衡状態における平均値だけでなく、熱や電荷の輸送などの動的現象、非平衡状態からの緩和現象などもシミュレーションできる
  \end{itemize}
\end{frame}

%\input{6_monte_carlo/molecular_dynamics-02.tex}
%\input{6_monte_carlo/molecular_dynamics-03.tex}

% \section{常微分方程式の初期値問題(復習)}
\input{2_ode/ode-01.tex}
\input{2_ode/ode-03.tex}
\input{2_ode/ode-07.tex}
\input{2_ode/ode-08.tex}

\section{シンプレクティック積分法}
\input{2_ode/symplectic-01.tex}
\input{2_ode/symplectic-02.tex}
\input{2_ode/symplectic-03.tex}
\input{2_ode/symplectic-04.tex}

%\section{ベルレ法}
\input{2_ode/symplectic-06.tex}
\input{2_ode/symplectic-07.tex}
\input{2_ode/symplectic-08.tex}

%\section{シンプレクティック積分法の一般論}
\input{2_ode/symplectic-09.tex}
\input{2_ode/symplectic-10.tex}
\input{2_ode/symplectic-11.tex}
\input{2_ode/symplectic-12.tex}
\begin{frame}[t,fragile]{シンプレクティック積分法}
  \begin{itemize}
    %\setlength{\itemsep}{1em}
  \item ハミルトン力学系の満たすべき特性(位相空間の体積保存)を満たす
  \item 一般的には陰解法
  \item ハミルトニアンが$H(p,q) = T(p) + V(q)$の形で書ける場合は陽的なシンプレクティック積分法が存在する
  \item エネルギーは近似的に保存する
  \item $n$次のシンプレクティック積分法では、エネルギーは$O(h^n)$の範囲で振動(発散しない)
  \item より高次のシンプレクティック積分法についても、システマティックに構成できる(ただし係数を解析的に求められるのは4次まで)。
    参考文献: H. Yoshida, Phys. Lett. A {\bf 150}, 262 (1990)
  \end{itemize}
  \begin{itemize}
    \item 系の満たすべき保存則を満たすように差分法を構成することが重要
    \item 参考: 計算科学フォーラムでの河合宗司先生(東北大)の講演 \url{https://hpcic-kkf.com/forum/2024/kkf_01/}
  \end{itemize}
\end{frame}


%\section{分子動力学法}

%%-*- coding:utf-8 -*-

\begin{frame}[t,fragile]{分子動力学法}
  \begin{itemize}
    \setlength{\itemsep}{1em}
  \item 適当な初期条件から、運動方程式に従って位置と運動量を時間発展させる
    \begin{itemize}
    \item Euler法、Runge-Kutta法、リープ・フロッグ法、速度ベルレ法など
    \item $6N$次元の連立微分方程式
    \end{itemize}
  \item 時間発展に関する物理量の時間平均から平均を評価
    \begin{align*}
      \langle A(p,x) \rangle &= \frac{1}{Z(E)} \int A(p,x) \, \delta(H(p,x)-E) \, dp \, dx \\
      &\simeq \frac{1}{t_{\rm max}} \int_0^{t_{\rm max}} A(p(t),x(t)) \, dt
    \end{align*}
    \begin{itemize}
    \item ハミルトニアンが時間依存しない場合は全エネルギーが保存する→ミクロカノニカル分布
    \end{itemize}
  \item 平衡状態における平均値だけでなく、熱や電荷の輸送などの動的現象、非平衡状態からの緩和現象などもシミュレーションできる
  \end{itemize}
\end{frame}

%\input{6_monte_carlo/molecular_dynamics-02.tex}

\section{長距離ポテンシャルの計算}

\input{6_monte_carlo/molecular_dynamics-03.tex}
\input{6_monte_carlo/longrange-01.tex}
\input{6_monte_carlo/longrange-02.tex}
\input{6_monte_carlo/longrange-03.tex}
\input{6_monte_carlo/longrange-04.tex}

\section{ビリアル定理}

\input{6_monte_carlo/virial-01.tex}
\input{6_monte_carlo/virial-02.tex}
%-*- coding:utf-8 -*-

\begin{frame}[t,fragile]{圧力・温度の計算}
  \begin{itemize}
    %\setlength{\itemsep}{1em}
  \item 壁から受ける力を考えると
    \begin{align*}
      \mathbf{f}_i &= \mathbf{f}_i^\text{int} + \mathbf{f}_i^\text{ext} = -\nabla_i U + \mathbf{f}_i^\text{ext} \\
      \sum_i \mathbf{f}_i^\text{ext} \cdot \mathbf{r}_i &= - \int_{\partial V} (P \mathbf{n}) \cdot \mathbf{r} \, dS = - P \int \nabla \cdot \mathbf{r} \, dV = -3PV
    \end{align*}
  \item ビリアル定理と組み合わせて
    \[
    PV = \frac{2}{3} \langle T \rangle -\frac{1}{3} \langle \sum_i \nabla_i U \cdot \mathbf{r}_i \rangle
    \]
  \item 温度の計算: エネルギー等分配則より
    \[
    \langle T \rangle = \frac{3}{2} k_B T N
    \]
    %\begin{itemize}
    %\item
    (全運動量を0に固定している場合は$\langle T \rangle = \frac{3}{2} k_B T (N-1)$)
    %\end{itemize}
  \end{itemize}
\end{frame}


\section{温度の制御}

\input{6_monte_carlo/molecular_dynamics-04.tex}
%-*- coding:utf-8 -*-

\begin{frame}[t,fragile]{温度の制御}
  \begin{itemize}
    %\setlength{\itemsep}{1em}
  \item ランジュバン(Langevin)法
    \begin{itemize}
    \item 乱数を使う方法
    \item 摩擦項と揺動項(ランダム力)を付け加える
      \begin{align*}
        \frac{dp_i}{dt} = - \frac{\partial H}{\partial q_i} - \gamma_i p_i + R_i(t)
      \end{align*}
    \item $\gamma_i$: 摩擦係数
    \item $R_i(T)$: 平均零の白色ノイズ
      \begin{align*}
        \langle R_i(t) R_i(t') \rangle = 2 m \gamma_i k_B T \delta(t-t')
      \end{align*}
    \item 摩擦項と揺動項がつりあうところで、温度$T$のカノニカル分布が実現する
    \end{itemize}
  \item Andersen法
    \begin{itemize}
    \item 各粒子と熱浴の結合係数: \(\nu\)
    \item 時間幅\(h\)の時間後、各粒子に対して確率\(\nu h\)でボルツマン分布から速度を再設定する。
    \end{itemize}
  \end{itemize}
\end{frame}

%-*- coding:utf-8 -*-

\begin{frame}[t,fragile]{Nose-Hoover熱浴}
  \begin{itemize}
    %\setlength{\itemsep}{1em}
  \item Nose (能勢)-Hoover法
    \begin{itemize}
    \item 熱浴をたった1つの自由度($s$)だけで実現する
    \item 現実系のハミルトニアン
      \begin{align*}
        H(\mathbf{p},\mathbf{x}) &= \sum_i \frac{p_i^2}{2m} + U(\mathbf{x})
      \end{align*}
    \item 仮想系のハミルトニアン (温度$T$をパラメータとして含む)
      \begin{align*}
        H'(\mathbf{p}',\mathbf{x}',{\color{red}p_s},{\color{red}s}) &= \sum_i \frac{{p'_i}^2}{2m{\color{red}s^2}} + U(\mathbf{x}') + {\color{red}\frac{p_s^2}{2Q} + g k_B T\log s}
      \end{align*}
      $s$: 熱浴の自由度、$p_s$: $s$に共役な運動量、$Q$: 熱浴の「質量」、$g$: 系の自由度($3N+1$または$3N$)
    \end{itemize}
  \end{itemize}
\end{frame}


\input{6_monte_carlo/molecular_dynamics-07.tex}
\input{6_monte_carlo/molecular_dynamics-08.tex}
%-*- coding:utf-8 -*-

\begin{frame}[t,fragile]{カノニカル分布の実現}
  \begin{itemize}
    %\setlength{\itemsep}{1em}
  \item $H'(\mathbf{p}',\mathbf{x}',p_s,s)$による仮想時間発展によりエネルギー$E'$のミクロカノニカルアンサンブルが実現しているとする

    $\Leftrightarrow$ 仮想時間$t'$でサンプルすると$2(3N+1)$次元の位相空間上で$H'=E'$の曲面上に均等に分布(エルゴード性)
    
    $\Leftrightarrow$ 分布関数: $\delta [ H'(\mathbf{p}',\mathbf{x}',p_s,s) - E']$
    
    $\Rightarrow$ 現実系の変数$(p,x)$に関する周辺分布を考える
  \item 仮想系のミクロカノニカル分配関数
    \begin{align*}
      Z'&=\int d\mathbf{p}' d\mathbf{x}' dp_s ds \, \delta [ H'(\mathbf{p}',\mathbf{x}',p_s,s) - E'] \\
      &=\int d\mathbf{{\color{red}p}} d\mathbf{{\color{red}x}} dp_s ds \, {\color{red}s^{3N}} \delta [ {\color{red}H(\mathbf{p},\mathbf{x})} + \frac{p_s^2}{2Q} + g k_B T \log s - E']
    \end{align*}
  \end{itemize}
\end{frame}

\input{6_monte_carlo/molecular_dynamics-10.tex}
\input{6_monte_carlo/molecular_dynamics-11.tex}
\input{6_monte_carlo/molecular_dynamics-12.tex}
\input{6_monte_carlo/molecular_dynamics-13.tex}
\input{6_monte_carlo/molecular_dynamics-14.tex}

% -*- coding: utf-8 -*-

\section{ハミルトニアン・モンテカルロ法}

%-*- coding:utf-8 -*-

\begin{frame}[t,fragile]{ハミルトニアン・モンテカルロ法}
  \begin{itemize}
    %\setlength{\itemsep}{1em}
  \item 通常のマルコフ連鎖モンテカルロ法
  \begin{itemize}
    \item 配位を局所的に変化させる ⇒ 配位空間上でランダムウォーク(拡散)
    \item 配位をむやみに大きく変化させようとしてもかえって状況は悪化
    \begin{itemize}
      \item cf. クラスターアルゴリズム, 拡張アンサンブル法
    \end{itemize}
  \end{itemize}

  \item ハミルトニアン・モンテカルロ法(HMC: Hamiltonian Monte Carlo)
  \begin{itemize}
    \item 別名: ハイブリッド・モンテカルロ法(Hybrid Monte Carlo)
    \item 分子動力学法 + メトロポリス法
    \item 仮想的な運動量を導入し、拡張した位相空間上でのマルコフ連鎖を考える
    \begin{itemize}
      \item 分子動力学法による仮想的な時間発展を用いて次の候補配位を生成
      \item 高い acceptance rate を保ちながら、大域的な配位変化を実現
    \end{itemize}
    \item 物性物理ではあまり使われていないが、Lattice QCDやベイズ推定ではよく使われている
    \end{itemize}
\end{itemize}
\end{frame}

%-*- coding:utf-8 -*-

\begin{frame}[t,fragile]{ハミルトニアン・モンテカルロ法が有効な例}
  \begin{itemize}
    %\setlength{\itemsep}{1em}
  \item 状態変数が連続変数の場合のみ使用可能
  \begin{itemize}
    \item 結合した調和振動子系
      \begin{align*}
        E(x_1,x_2,\cdots,x_N) = \sum x_i^2 + \sum_{i,j} C_{i,j} (x_i - x_j)^2
      \end{align*}
    \item 連続古典スピン系 (ハイゼンベルグ模型)
      \begin{align*}
        E(\vec{S}_1, \vec{S}_2, \cdots, \vec{S}_N) = \sum_{i,j} J_{i,j} \vec{S}_i \cdot \vec{S}_j
      \end{align*}
  \end{itemize}

  \item エネルギーの状態変数に関する微分の情報(=力)を利用する
  \item 現実の系の時間発展と同一である必要はない - 仮想的な時間発展
\end{itemize}
\end{frame}

%-*- coding:utf-8 -*-

\begin{frame}[t,fragile]{ハミルトン力学系}
  \begin{itemize}
    %\setlength{\itemsep}{1em}
  \item もともとの状態変数: \(\vec{q} = (q_1,q_2,\cdots,q_N)\)
  \item ボルツマン重み(逆温度\(\beta\)は\(E\)に含む): \(P(\vec{q}) = e^{-E(\vec{q})}\)
  \item 状態変数の「時間発展」を考える
  \item 「運動量」「運動エネルギー」「ハミルトニアン」を導入
    \begin{align*}
      \vec{p} &= \frac{d\vec{q}}{dt} = (\vec{p}_1(t),\vec{p}_2(t),\cdots,\vec{p}_N(t)) \\
      H(\vec{q}, \vec{p}) &= K(\vec{p}) + E(\vec{q}) = \frac{1}{2}\sum_i p_i^2 + E(\vec{q})
    \end{align*}
  \item 「ハミルトン力学系」の「時間発展」
    \begin{align*}
      \begin{cases}
        \displaystyle
        \frac{dq_i}{dt} = \frac{\partial H}{\partial p_i} \\
        \displaystyle
        \frac{dp_i}{dt} = -\frac{\partial H}{\partial q_i}
      \end{cases}
    \end{align*}
\end{itemize}
\end{frame}

%-*- coding:utf-8 -*-

\begin{frame}[t,fragile]{ハミルトン力学系}
  \begin{itemize}
    %\setlength{\itemsep}{1em}
  \item 「時間発展」の特徴
  \begin{itemize}
    \item 「全エネルギー」が保存
    \[
      \frac{dH}{dt} = \frac{\partial H}{\partial q} \frac{dq}{dt} + \frac{\partial H}{\partial p} \frac{dp}{dt} = 0
    \]
    \item 位相空間の体積が保存(Liouvilleの定理)

          位相空間上の流れの場\(\bm{v} = (\frac{dq}{dt},\frac{dp}{dt})\)について
          \[
          \text{div} \, \bm{v} = \frac{\partial}{\partial q} \frac{dq}{dt} + \frac{\partial}{\partial p} \frac{dp}{dt} = 0
          \]
          \end{itemize}
  \item 位相空間上の同時分布が\(P(\vec{q},\vec{p}) \sim \exp(-H(\vec{q},\vec{p}))\)形の時
  \begin{itemize}
    \item \(P(\vec{q},\vec{p})\)は時間発展の下で不変
    \item 我々の欲しいボルツマン分布は\(P(\vec{q},\vec{p})\)の周辺分布
  \end{itemize}
\end{itemize}
\end{frame}

%-*- coding:utf-8 -*-

\begin{frame}[t,fragile]{位相空間での状態更新}
  \begin{itemize}
    %\setlength{\itemsep}{1em}
  \item ハミルトン方程式に従って適当な時間\(\tau\)だけ時間発展を行う
  \item 位置\(\vec{q}\)を固定して運動量\(\vec{p}\)だけを確率的に更新 (「全エネルギー」を変更 )
  \begin{itemize}
    \item \(p_i\)の条件付き確率 (正規分布 \(\mathcal{N}(0,1)\))
    \begin{align*}
      p(p_i|\vec{q},p_1,\cdots,p_{i-1},p_{i+1},\cdots,p_N) &\sim \exp (-\frac{1}{2}\sum p_j^2+E(\vec{q})) \\ &\sim \exp(-\frac{1}{2}p_i^2)
    \end{align*}
  \end{itemize}

  \item 実際には差分法を用いて運動方程式を積分するので有限の離散誤差
  \begin{itemize}
    \item Liouvilleの定理を厳密に満たす差分法(リープフロッグ)を用いる
    \item 分子動力学法 + メトロポリス法
    \item 「全エネルギー」の値がずれた分をメトロポリス法で修正
  \end{itemize}

  \item 詳細ついあい条件
  \begin{align*}
    &\exp(-E(\vec{q})) \exp(-K(\vec{p})) P((\vec{q},\vec{p})\rightarrow(\vec{q'},\vec{p'})) \min (1, \exp(H-H')) \\
    & = \exp(-E(\vec{q'})) \exp(-K(\vec{p'})) P((\vec{q'},\vec{p'})\rightarrow(\vec{q},\vec{p})) \min (1, \exp(H'-H))
  \end{align*}
\end{itemize}
\end{frame}

%-*- coding:utf-8 -*-

\begin{frame}[t,fragile]{ハミルトニアン・モンテカルロ法}
  \begin{itemize}
    %\setlength{\itemsep}{1em}
    \item 分子動力学法 + メトロポリス法
    \item 仮想的な運動量を導入し、拡張した位相空間上でのマルコフ連鎖を考える
    \item 分子動力学法による仮想的な時間発展を用いて次の候補配位を生成
    \begin{itemize}
      \item 位相空間の体積を保存する差分法を用いることにより、提案確率が対称になる
    \end{itemize}
    \item 高い acceptance rate を保ちながら、大域的な配位変化を実現
    \begin{itemize}
      \item エネルギーをさらに良く保存する高い次数の差分法を用いれば、より acceptance rate が高くなる
    \end{itemize}
    \item すばやくかき混ぜるというアルゴリズム的な目標を達成するために、仮想的に「物理」を導入
\end{itemize}
\end{frame}



\end{document}
